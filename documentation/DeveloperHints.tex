\documentclass[a4paper]{article}

\usepackage[utf8]{inputenc}
\usepackage[english]{babel}
\usepackage{hyperref}

\begin{document}

\section*{Install opencl for Intel}
Run the following in this order:
\begin{itemize}
\item  Generic ubuntu packages for OpenCL:
  \begin{itemize}
  \item sudo apt install ocl-icd-libopencl1
  \item sudo apt install opencl-headers
  \item sudo apt install clinfo
  \end{itemize}
\item Package that allows to compile OpenCL code:
   \begin{itemize}
   \item sudo apt install ocl-icd-opencl-dev
   \end{itemize}
\item For Intel GT core (e.g: us):
  \begin{itemize}
   \item sudo apt install beignet
   \end{itemize}
 \end{itemize}
NOTE: for opencl with NVIDIA, it is enough to install the driver.

\section*{OpenCL 2.0 with NVIDIA cards}
How to run opencl 2.0 kernels with an NVIDIA graphics card:

\begin{itemize}
\item Checking clinfo (or anything else) will show that your NVIDIA device only supports
  opencl 2.0.
\item But, you can still compile a kernel with opencl 2.0: when
  calling \texttt{cl::Program::build(...)} or \texttt{clBuildProgram(...)}, pass an additional
  \texttt{const char*} argument (which contains the build flags):
  \texttt{``-cl-std=CL2.0''}.
\item Note, that nvidia has limited opencl 2.0 support... the
  following is a thread from 2017 (\href{https://streamhpc.com/blog/2017-02-22/nvidia-enables-opencl-2-0-beta-support/}{link}): \newline
  ``New features in OpenCL 2.0 are available in the driver for evaluation purposes only. The following are the features as well as a description of known issues in the driver:
  \begin{itemize}
  \item Device side enqueue
    \begin{itemize}
    \item The current implementation is limited to 64-bit platforms only.
    \item OpenCL 2.0 allows kernels to be enqueued with
      global\_work\_size larger than the compute capability of the
      NVIDIA GPU. The current implementation supports only
      combinations of global\_work\_size and local\_work\_size that are
      within the compute capability of the NVIDIA GPU. The maximum
      supported CUDA grid and block size of NVIDIA GPUs is available
      at
      \href{http://docs.nvidia.com/cuda/cuda-c-programming-guide/index.html#computecapabilities}{this link}. For
      a given grid dimension, the global\_work\_size can be determined
      by CUDA grid size x CUDA block size.
    \item For executing kernels (whether from the host or the
      device), OpenCL 2.0 supports non-uniform ND-ranges where
      global\_work\_size does not need to be divisible by the
      local\_work\_size. \textbf{This capability is not yet supported in the
        NVIDIA driver}, and therefore not supported for device side
      kernel enqueues.
    \end{itemize}
  \item Shared virtual memory: The current implementation of shared
    virtual memory is limited to 64-bit platforms only.''
  \end{itemize}
\end{itemize}


\end{document}